%%%%%%%%%%%%%%%%%%%%%%%%%%%
%%% PART 1 - CHAPITRE 1 %%%
%%%%%%%%%%%%%%%%%%%%%%%%%%%

\chapter{Présentation de \LaTeX}
\section{\LaTeX, c'est...}
\LaTeX{} (se prononce \textit{latek}) est un langage créé originellement par des scientifiques qui cherchaient à rédiger des documents capables de gérer les mises en forme d'expression mathématiques, tout en offrant la possibilité d'y ajouter des extensions. Cette spécificité liée aux écritures mathématiques a rendu \LaTeX{} populaire parmi la communauté scientifique.

\begin{figure}[h]
\begin{center}
\includegraphics[scale=0.3]{IMG/LogoLaTex.jpeg}
\caption{Logo de \LaTeX}
\end{center}
\end{figure}

\LaTeX{} est un langage de balisage qu'il est aisé d'apprendre, et s'avère très utile. Au regard du rendu et de l'élégance des documents produits (mise en page de façon professionnelle), il offre de larges possibilités pour la rédaction d'articles, de mémoires, de thèses, mais aussi de livres. C'est en fait un langage de description qui permet de respecter les normes éditoriales et typographiques.
\medskip

Comparativement aux éditeurs de texte plus connus, tel que \textit{LibreOffice}, \LaTeX{} se montre bien plus efficace et aisé d'utilisation pour:
\begin{itemize}
\item la modification des styles de titres;
\item la gestion des notes;
\item la gestion des flottants (les figures que l'on insère dans les documents);
\item la rédaction et la gestion des longs documents;
\item la hiérarchisation du texte (parties, chapitres, sections, etc.);
\item la gestion des références internes au document;
\item la gestion des bibliographies, index et tables des matières.
\end{itemize}
\medskip

Avec \LaTeX{} tout est modifiable et paramétrable. De plus, nous pouvons à partir d'un document rédigé en \LaTeX de générer, par une simple compilation du fichier source \texttt{.tex}, des documents en \texttt{PDF}.
\medskip

\section{La petite histoire de \LaTeX}
Tout débute en 1977 par la création du langage \TeX{} par Donald Erwin \textsc{Knuth} qui cherchait alors à accroître la lisibilité et à optimiser l'insertion des formules mathématiques dans les articles scientifiques.
\medskip

En 1985, Leslie \textsc{Lamport} créé \LaTeX, bien plus aisé à utiliser que \TeX{}. Diverses tâches sont alors simplifiées à l'aide de macros intégrées au programme. Par la suite, suite à une évolution majeure, ce sera \LaTeXe{} qui sera majoritaire utilisé.
\medskip
