%%%%%%%%%%%%%%%%%%%%%%%%%%%
%%% PART 1 - CHAPITRE 2 %%%
%%%%%%%%%%%%%%%%%%%%%%%%%%%

\chapter{Installer \LaTeX}
Une installation fonctionnelle comporte trois éléments:
\medskip

\begin{description}
\item[Une distribution \LaTeX]: le logiciel comprenant toutes les composantes nécessaires de \LaTeX.
\item[Un lecteur \texttt{PostScript} et/ou \texttt{PDF}]: afin de lire ses productions.
\item[Un éditeur \LaTeX]: qui facilite la rédaction d'un document rédigé en \LaTeX{}, même si un éditeur plus classique (exemple: \textit{Vim}) peut suffire.
\end{description}
\medskip

\section*{Installation sous \textit{Debian GNU/Linux}}
Pour la distribution \LaTeX{} on peut choisir d'installer le paquet \textit{texlive} (distribution de base) ou le paquet \textit{texlive-full} (distribution plus complète). On ajoutera le paquet \textit{cm-super} afin de disposer de polices supplémentaires.
\medskip

Pour lire et manipuler les fichiers \texttt{.ps} on installera \textit{gv}, et pour les fichiers \texttt{PDF} le lecteur \textit{evince} (bien intégré au bureau \textit{Gnome} sera suffisant.
\medskip

Enfin, concerant l'éditeur \LaTeX{} cela est souvent une question de goût et/ou d'habitude personnelle. Pour ma part ma préférence va au logiciel \textit{Texmaker}, qui dispose en plus d'une interface en français.
\medskip
