%%%%%%%%%%%%%%%%%%%%%%%%%%%%
%%% PART 2 - CHAPITRE 11 %%%
%%%%%%%%%%%%%%%%%%%%%%%%%%%%

\chapter{Sommaire et index}
\section{Table des matières}
\subsection*{Table des matières simple}
Il suffit de placer la commande \verb|\tableofcontents| là où vous souhaitez insérer la table des matières. La table est conçue telle que le prévoient les paramètres par défaut de \LaTeX{}. cela nécessite deux compilations: la première permet à \LaTeX{} de comprendre la structure du document et de lister les titres, et la seconde permet la mise en forme de la table avec les numéros de pages.
\medskip

\subsection*{Paramétrage d'une table des matières}
\paragraph*{Appeler la table \og sommaire\fg{}}
avec la commande suivante à insérer dans le corps du document, juste avant la commande \verb|\tableofcontents|:
\begin{verbatim}
    \renewcommand{\contentsname}{Sommaire}
\end{verbatim}
\medskip

\paragraph*{Raccourcir une ligne}
Lors de la création d'un élément de structure il est possible de définir deux titres: l'un pour le document, l'autre pour la table des matières. On code cela de la manière suivante:
\begin{verbatim}
    \section{titre table des matières}{titre dans le document}
\end{verbatim}
\medskip

\paragraph*{Quel niveau hiérarchique dans notre table des matières ?}
Il est possible de définir le niveau hiérarchique de la table des matières (inclusion des sous-sections ou non, des sections, etc.). La commande à placer en préambule est alors la suivante:
\begin{verbatim}
    \setcounter{tocdepth}{Choix du niveau (nombre)}
\end{verbatim}
\medskip

Concernant le niveau hiérarchique choisi, voir le tableau suivant:
\begin{table}{!h}
\begin{center}
\begin{tabular}{|c|c|}
\hline
\textbf{Niveau hiérarchique (inclus)} & \textbf{Valeur} \\
\hline
Partie & -1 \\
\hline
Chapitre & 0 \\
\hline
Section & 1 \\
\hline
Sous-section & 2 \\
\hline
Sous-sous-section & 3 \\
\hline
Paragraphe & 4 \\
\hline
Sous-paragraphe & 5 \\
\hline
\end{tabular}
\caption{Tableau des niveaux hiérarchiques de la table des matières}
\end{center}
\end{table}
\medskip

A noter qu'avec la classe \texttt{book}, par défaut, les titres des paragraphes ne sont pas inclus.
\medskip

\section{Tables de figures et tableaux}
Ces tables suivent à peu prés les mêmes mécanismes que ceux de la table des matières. On utilise respectivement les commandes \verb|\listoffigures| et \verb|\listoftables|. Apparaissent dans ces tables, soit un titre spécialement conçu pour la table, soit la légende contenue dans la commande \verb|\caption{}|. Voici les deux usages de la commande \verb|\caption{}|:
\begin{verbatim}
    \caption{légende de la figure ou de la table}
    \caption[légende courte]{légende de la figure ou de la table}
\end{verbatim}
\medskip

Là encore, deux compilations sont nécessaires.
\medskip

\section{Les index}
La création d'un index nécessite le package \textit{makeindex} et l'insertion de la commande \verb|\makeindex| dans le préambule, puis on insère la commande \verb|\printindex| à l'endroit où l'on décide de l'afficher. Les entrées d'index seront marqués à l'aide de la commande \verb|\index{argument}| collée au mot que l'on souhaite indexé. L'utilisation d'index nécessite trois compilations. Dans l'index les \textit{arguments} figureront dans l'ordre alphabétique.
\medskip

Nous venons de voir la méthode la plus simple pour la création d'index, mais créer de véritables index s'avère beaucoup plus complexe.
\medskip

Commande permettant de d'insérer un \textit{argument} d'index avec une référence croisée:
\begin{verbatim}
    \index{argument|see{référence croisée}}
\end{verbatim}
\medskip
