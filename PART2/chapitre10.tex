%%%%%%%%%%%%%%%%%%%%%%%%%%%%
%%% PART 2 - CHAPITRE 10 %%%
%%%%%%%%%%%%%%%%%%%%%%%%%%%%

\chapter{Les tableaux}
Il faut savoir que \LaTeX{} considère les tableaux comme des objets flottants. Les tableaux font l'objet d'une documentation extrêmement fournie, à l'instar de la documentation des notations mathématiques. Dans ce chapitre nous allons juste nous contenter d'apprendre à:
\begin{itemize}
	\item composer des tableaux simples;
	\item fusionner des cellules;
	\item paramétrer le placement allié à quelques détails de mise en page.
\end{itemize} 
\medskip

\section{Structure type d'un tableau}
\subsection*{Tableau sans bordure}
Nous allons utiliser l'environnement \texttt{tabular}. Dans un premier temps il faut décider de l'alignement des cellules dans chaque colonne (options):
\begin{description}
	\item[r]: à droite;
	\item[l]: à gauche;
	\item[c]: centré.
\end{description}
\medskip

Exemple de squelette de tableau à deux colonnes:
\begin{verbatim}
    \begin{tabular}{cc}
        ...code pour mon tableau...
    \end{tabular}
\end{verbatim}
\medskip

Nous entrons ensuite ligne par ligne le contenu des cellules, séparées par le caractère \texttt{\&}. Chaque ligne se termine par \textbackslash \textbackslash{} pour indiquer que nous changeons de ligne.
\medskip

Exemple de tableau à deux colonnes et deux lignes:
\begin{verbatim}
    \begin{tabular}{cc}
        lig. 1 col. 1 & lig. 1 col. 2 \\
        lig. 2 col. 1 & lig. 2 col. 2 \\
    \end{tabular}
\end{verbatim}
\medskip

Ce qui donne:
\begin{figure}[!h]
\begin{center}
\begin{tabular}{cc}
        lig. 1 col. 1 & lig. 1 col. 2 \\
        lig. 2 col. 1 & lig. 2 col. 2 \\
\end{tabular}
\end{center}
\end{figure}
\medskip

\subsection*{Tableau avec bordures}
Pour obtenir une ligne horizontale on va utiliser la commande \verb|\hline|. Pour marquer les colonnes, on va utiliser \og \texttt{|}\fg{} lors de la spécification des alignements.
\medskip

Exemple d'une tableau à deux colonnes et deux lignes:
\begin{verbatim}
    \begin{tabular}{|c|c|}
        \hline
        lig. 1 col. 1 & lig. 1 col. 2 \\
        \hline
        lig. 2 col. 1 & lig. 2 col. 2 \\
        \hline
    \end{tabular}
\end{verbatim}
\medskip

Ce qui donne:
\begin{figure}[!h]
\begin{center}
\begin{tabular}{|c|c|}
\hline
lig. 1 col. 1 & lig. 1 col. 2 \\
\hline
lig. 2 col. 1 & lig. 2 col. 2 \\
\hline
\end{tabular}
\end{center}
\end{figure}
\medskip

\section{Fusion des cellules}
\subsection*{Fusion de colonnes}
La commande est la suivante:
\begin{verbatim}
    \multicolumn{nbre_colonnes_fusionnées}{alignements}{Texte}
\end{verbatim}
\medskip

La difficulté réside dans le choix un nouvel alignement pour la cellule fusionnée.
\medskip

Illustrons cela:
\begin{verbatim}
    \begin{tabular}{|c|c|c|c|c|c|c|c|c|c|c|}
    \hline
    Multiplié par & 1 & 2 & 3 & 4 & 5 & 6 & 7 & 8 & 9 & 10 \\
    \hline
    1 & 1 & 2 & 3 & 4 & 5 & 6 & 7 & 8 & 9 & 10 \\
    \hline
    2 & 2 & 4 & 6 & 8 & 10 & 12 & 14 & 16 & 18 & 20 \\
    \hline
    3 & 3 & 6 & 9 & 12 & 15 & 18 & 21 & 24 & 27 & 30 \\
    \hline
    4 & 4 & 8 & 12 & 16 & 20 & 24 & 28 & 32 & 36 & 40 \\
    \hline
    5 & 5 & 10 & 15 & 20 & 25 & 30 & 35 & 40 & 45 & 50 \\
    \hline
    6 & 6 & 12 & 18 & 24 & 30 & 36 & 42 & 48 & 54 & 60 \\
    \hline
    7 & 7 & 14 & 21 & 28 & 35 & 42 & 49 & 56 & 63 & 70 \\
    \hline
    8 & 8 & 16 & 24 & 32 & 40 & 48 & 56 & 64 & 72 & 80 \\
    \hline
    9 & 9 & 18 & 27 & 36 & 45 & 54 & 63 & 72 & 81 & 90 \\
    \hline
    10 & 10 & 20 & 30 & 40 & 50 & 60 & 70 & 80 & 90 & 100 \\
    \hline 
\end{tabular}
\end{verbatim}
\begin{table}[!h]
\begin{center}
\begin{tabular}{|c|c|c|c|c|c|c|c|c|c|c|}
\hline
Multiplié par & 1 & 2 & 3 & 4 & 5 & 6 & 7 & 8 & 9 & 10 \\
\hline
1 & 1 & 2 & 3 & 4 & 5 & 6 & 7 & 8 & 9 & 10 \\
\hline
2 & 2 & 4 & 6 & 8 & 10 & 12 & 14 & 16 & 18 & 20 \\
\hline
3 & 3 & 6 & 9 & 12 & 15 & 18 & 21 & 24 & 27 & 30 \\
\hline
4 & 4 & 8 & 12 & 16 & 20 & 24 & 28 & 32 & 36 & 40 \\
\hline
5 & 5 & 10 & 15 & 20 & 25 & 30 & 35 & 40 & 45 & 50 \\
\hline
6 & 6 & 12 & 18 & 24 & 30 & 36 & 42 & 48 & 54 & 60 \\
\hline
7 & 7 & 14 & 21 & 28 & 35 & 42 & 49 & 56 & 63 & 70 \\
\hline
8 & 8 & 16 & 24 & 32 & 40 & 48 & 56 & 64 & 72 & 80 \\
\hline
9 & 9 & 18 & 27 & 36 & 45 & 54 & 63 & 72 & 81 & 90 \\
\hline
10 & 10 & 20 & 30 & 40 & 50 & 60 & 70 & 80 & 90 & 100 \\
\hline
\end{tabular}
\caption{Table de multiplication}
\end{center}
\end{table}
\medskip

\subsection*{Fusion de lignes}
Pour la fusion de lignes c'est la commande \verb|\multirow| contenue dans le package du même nom.
\begin{verbatim}
    \multirow{nbre de lignes fusionnées}{taille colonne en cm}{texte}
    \multirow{nbre de lignes fusionnées}*{texte}
\end{verbatim}

Observez bien le code suivant et le résultat qui s'en suit:
\begin{verbatim}
    \begin{tabular}{|l|c|c|c|c|}
    \hline
    1 & \multicolumn{2}{c|}{2} & 3 & 4 \\
    \hline
    \multicolumn{2}{|l|}{5} & 6 & 7 & 8 \\
    \hline
    9 & 10 & \multicolumn{3}{c|}{11} \\
    \hline
    \multirow{2}{1cm}{12} & 13 & 14 & 15 & 16 \\
    \cline{2-5}
    & 17 & 18 & 19 & 20 \\
    \hline
    21 & 22 & \multirow{2}*{23} & 24 & 25 \\
    \cline{1-2} \cline{4-5}
    26 & 27 & & 28 & 29 \\
    \hline
    \end{tabular}
\end{verbatim}
\medskip

\begin{table}[!h]
\begin{center}
\begin{tabular}{|l|c|c|c|c|}
\hline
1 & \multicolumn{2}{c|}{2} & 3 & 4 \\
\hline
\multicolumn{2}{|l|}{5} & 6 & 7 & 8 \\
\hline
9 & 10 & \multicolumn{3}{c|}{11} \\
\hline
\multirow{2}{1cm}{12} & 13 & 14 & 15 & 16 \\
\cline{2-5}
& 17 & 18 & 19 & 20 \\
\hline
21 & 22 & \multirow{2}*{23} & 24 & 25 \\
\cline{1-2} \cline{4-5}
26 & 27 & & 28 & 29 \\
\hline
\end{tabular}
\caption{Fusion de lignes et de colonnes}
\end{center}
\end{table}
\medskip

\section{Autres paramètres applicables à un tableau}
\subsection*{Colonne de largeur paramétrée}
\begin{verbatim}
    p{largeur de la colonne en centimètres}
\end{verbatim}
\medskip

Cette option n'a aucune influence sur l'alignement du texte au sein des cellules.
\medskip

Un exemple:
\begin{verbatim}
    \begin{tabular}{|p{1cm}|p{2cm}|p{3cm}|p{4cm}|}
    \hline
    1cm & 2cm & 3cm & 4cm \\
    \hline
    \end{tabular}
\end{verbatim}
\medskip

Ce qui donne:
\begin{table}[!h]
\begin{center}
\begin{tabular}{|p{1cm}|p{2cm}|p{3cm}|p{4cm}|}
    \hline
    1cm & 2cm & 3cm & 4cm \\
    \hline
\end{tabular}
\caption{Cellules de longueur définie}
\end{center}
\end{table}
\medskip

\subsection*{Créer une \textit{slashbox}}
Le package \textit{slashbox} permet d'utiliser la commande:
\begin{verbatim}
    \backslashbox{Texte dessous}{Texte dessus}
\end{verbatim}
\medskip

Cette commande sert à scinder en deux parties triangulaires de même aire une cellule initialement rectangulaire.
\medskip

Démonstration:
\begin{verbatim}
    \begin{tabular}{|c|p{1cm}|p{2cm}|}
    \hline
    \backslashbox{Debian}{Ubuntu} & 1cm & 2cm \\
    \hline
    \end{tabular}
\end{verbatim}
\medskip

Résultat:
\begin{table}
\begin{center}
\begin{tabular}{|c|p{1cm}|p{2cm}|}
\hline
\backslashbox{Debian}{Ubuntu} & 1cm & 2cm \\
\hline
\end{tabular}
\caption{Avec le package \textit{slashbox}}
\end{center}
\end{table}
\medskip

\subsection*{Changer les séparateurs}
Cela est possible à l'aide des commandes \verb|!{séparateur}| ou \verb|@{séparateur}|, contenues dans le package \textit{array}. A noter que ce package contient beaucoup de commandes utiles à la création des tableaux. La différence entre ces deux commandes réside dans le fait de pouvoir insérer une espace avant et après le séparateur.
\begin{verbatim}
    \begin{tabular}{|c !{\#} c @{\#}c|}
    \hline
    3 + 1 & = & 4 \\
    \hline
    \end{tabular}
\end{verbatim}
\medskip

Ce qui a pour résultat:
\begin{table}[!h]
\begin{center}
\begin{tabular}{|c !{\#} c @{\#}c|}
\hline
3 + 1 & = & 4 \\
\hline
\end{tabular}
\caption{De nouveaux séparateurs}
\end{center}
\end{table}
\medskip

\section{Des commandes et des environnements dans un tableau}
Démonstration avec la table de multiplication:
\begin{verbatim}
    \begin{tabular}{|>{\begin{bf}} c <{\end{bf}}|c|c|c|c|c|c|c|c|c|c|}
 
    \hline
    Multiplié par & \begin{bf}1\end{bf} & \begin{bf}2\end{bf} & 
    	\begin{bf}3\end{bf} & \begin{bf}4\end{bf} & \begin{bf}5\end{bf} & 
    	\begin{bf}6\end{bf} & \begin{bf}7\end{bf} & \begin{bf}8\end{bf} & 
    	\begin{bf}9\end{bf} & \begin{bf}10\end{bf} \\
    \hline
    1 & 1 & 2 & 3 & 4 & 5 & 6 & 7 & 8 & 9 & 10 \\
    \hline
    2 & 2 & 4 & 6 & 8 & 10 & 12 & 14 & 16 & 18 & 20 \\
    \hline
    3 & 3 & 6 & 9 & 12 & 15 & 18 & 21 & 24 & 27 & 30 \\
    \hline
    4 & 4 & 8 & 12 & 16 & 20 & 24 & 28 & 32 & 36 & 40 \\
    \hline
    5 & 5 & 10 & 15 & 20 & 25 & 30 & 35 & 40 & 45 & 50 \\
    \hline
    6 & 6 & 12 & 18 & 24 & 30 & 36 & 42 & 48 & 54 & 60 \\
    \hline
    7 & 7 & 14 & 21 & 28 & 35 & 42 & 49 & 56 & 63 & 70 \\
    \hline
    8 & 8 & 16 & 24 & 32 & 40 & 48 & 56 & 64 & 72 & 80 \\
    \hline
    9 & 9 & 18 & 27 & 36 & 45 & 54 & 63 & 72 & 81 & 90 \\
    \hline
    10 & 10 & 20 & 30 & 40 & 50 & 60 & 70 & 80 & 90 & 100 \\
    \hline
 
    \end{tabular}
\end{verbatim}
\medskip

Affichage:
\begin{table}[!h]
\begin{center}
\begin{tabular}{|>{\begin{bf}} c <{\end{bf}}|c|c|c|c|c|c|c|c|c|c|}
 
\hline
Multiplié par & \begin{bf}1\end{bf} & \begin{bf}2\end{bf} & \begin{bf}3\end{bf} & \begin{bf}4\end{bf} & \begin{bf}5\end{bf} & \begin{bf}6\end{bf} & \begin{bf}7\end{bf} & \begin{bf}8\end{bf} & \begin{bf}9\end{bf} & \begin{bf}10\end{bf} \\
\hline
1 & 1 & 2 & 3 & 4 & 5 & 6 & 7 & 8 & 9 & 10 \\
\hline
2 & 2 & 4 & 6 & 8 & 10 & 12 & 14 & 16 & 18 & 20 \\
\hline
3 & 3 & 6 & 9 & 12 & 15 & 18 & 21 & 24 & 27 & 30 \\
\hline
4 & 4 & 8 & 12 & 16 & 20 & 24 & 28 & 32 & 36 & 40 \\
\hline
5 & 5 & 10 & 15 & 20 & 25 & 30 & 35 & 40 & 45 & 50 \\
\hline
6 & 6 & 12 & 18 & 24 & 30 & 36 & 42 & 48 & 54 & 60 \\
\hline
7 & 7 & 14 & 21 & 28 & 35 & 42 & 49 & 56 & 63 & 70 \\
\hline
8 & 8 & 16 & 24 & 32 & 40 & 48 & 56 & 64 & 72 & 80 \\
\hline
9 & 9 & 18 & 27 & 36 & 45 & 54 & 63 & 72 & 81 & 90 \\
\hline
10 & 10 & 20 & 30 & 40 & 50 & 60 & 70 & 80 & 90 & 100 \\
\hline
 
\end{tabular}
\caption{Mise en gras de certaines parties}
\end{center}
\end{table}
\medskip

\subsection*{Colorer des cellules}
Pour cela deux packages sont nécessaires: \textit{color} et \textit{colortbl}. Les commandes sont les suivantes:
\begin{verbatim}
    \columncolor{couleur}  % Pour colorer les colonnes
    \rowcolor{couleur}  % Pour colorer des lignes
    \cellcolor{couleur}  % Pour colorer les cellules
\end{verbatim}
\medskip

Exemple d'un tableau avec la première ligne et la première colonne sur fond jaune:
\begin{verbatim}
    \begin{tabular}{>{\begin{bf} \columncolor{yellow}} c <{\end{bf}}cccccccccc}
    \hline  
    \rowcolor{yellow}Multiplié par & \begin{bf}1\end{bf} & \begin{bf}2\end{bf} & 
        \begin{bf}3\end{bf} & \begin{bf}4\end{bf} & \begin{bf}5\end{bf} & 
        \begin{bf}6\end{bf} & \begin{bf}7\end{bf} & \begin{bf}8\end{bf} & 
        \begin{bf}9\end{bf} & \begin{bf}10\end{bf} \\
    \hline 
    1 & 1 & 2 & 3 & 4 & 5 & 6 & 7 & 8 & 9 & 10 \\
    \hline
    2 & 2 & 4 & 6 & 8 & 10 & 12 & 14 & 16 & 18 & 20 \\
    \hline
    3 & 3 & 6 & 9 & 12 & 15 & 18 & 21 & 24 & 27 & 30 \\
    \hline
    4 & 4 & 8 & 12 & 16 & 20 & 24 & 28 & 32 & 36 & 40 \\
    \hline
    5 & 5 & 10 & 15 & 20 & 25 & 30 & 35 & 40 & 45 & 50 \\
    \hline
    6 & 6 & 12 & 18 & 24 & 30 & 36 & 42 & 48 & 54 & 60 \\
    \hline
    7 & 7 & 14 & 21 & 28 & 35 & 42 & 49 & 56 & 63 & 70 \\
    \hline
    8 & 8 & 16 & 24 & 32 & 40 & 48 & 56 & 64 & 72 & 80 \\
    \hline
    9 & 9 & 18 & 27 & 36 & 45 & 54 & 63 & 72 & 81 & 90 \\
    \hline
    10 & 10 & 20 & 30 & 40 & 50 & 60 & 70 & 80 & 90 & 100 \\
    \hline
    \end{tabular}
\end{verbatim}
\medskip

Affichage:
\begin{table}[!h]
\begin{center}
\begin{tabular}{>{\begin{bf} \columncolor{yellow}} c <{\end{bf}}cccccccccc}
\hline  
\rowcolor{yellow}Multiplié par & \begin{bf}1\end{bf} & \begin{bf}2\end{bf} & \begin{bf}3\end{bf} & \begin{bf}4\end{bf} & \begin{bf}5\end{bf} & \begin{bf}6\end{bf} & \begin{bf}7\end{bf} & \begin{bf}8\end{bf} & \begin{bf}9\end{bf} & \begin{bf}10\end{bf} \\
\hline 
1 & 1 & 2 & 3 & 4 & 5 & 6 & 7 & 8 & 9 & 10 \\
\hline
2 & 2 & 4 & 6 & 8 & 10 & 12 & 14 & 16 & 18 & 20 \\
\hline
3 & 3 & 6 & 9 & 12 & 15 & 18 & 21 & 24 & 27 & 30 \\
\hline
4 & 4 & 8 & 12 & 16 & 20 & 24 & 28 & 32 & 36 & 40 \\
\hline
5 & 5 & 10 & 15 & 20 & 25 & 30 & 35 & 40 & 45 & 50 \\
\hline
6 & 6 & 12 & 18 & 24 & 30 & 36 & 42 & 48 & 54 & 60 \\
\hline
7 & 7 & 14 & 21 & 28 & 35 & 42 & 49 & 56 & 63 & 70 \\
\hline
8 & 8 & 16 & 24 & 32 & 40 & 48 & 56 & 64 & 72 & 80 \\
\hline
9 & 9 & 18 & 27 & 36 & 45 & 54 & 63 & 72 & 81 & 90 \\
\hline
10 & 10 & 20 & 30 & 40 & 50 & 60 & 70 & 80 & 90 & 100 \\
\hline
\end{tabular}
\caption{Tableau avec des couleurs}
\end{center}
\end{table}
\medskip

\subsection*{L'environnement flottant \texttt{table}}
Cela revient à insérer l'environnement \texttt{tabular} dans un environnement flottant. Celui-ci se nomme \texttt{table} et est similaire en tout point à l'environnement \texttt{figure}, avec l'utilisation de \texttt{caption}, \texttt{label}, \texttt{center}, etc.
\medskip
